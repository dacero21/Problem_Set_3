% Options for packages loaded elsewhere
\PassOptionsToPackage{unicode}{hyperref}
\PassOptionsToPackage{hyphens}{url}
%
\documentclass[
  12pt,
]{article}
\usepackage{amsmath,amssymb}
\usepackage{iftex}
\ifPDFTeX
  \usepackage[T1]{fontenc}
  \usepackage[utf8]{inputenc}
  \usepackage{textcomp} % provide euro and other symbols
\else % if luatex or xetex
  \usepackage{unicode-math} % this also loads fontspec
  \defaultfontfeatures{Scale=MatchLowercase}
  \defaultfontfeatures[\rmfamily]{Ligatures=TeX,Scale=1}
\fi
\usepackage{lmodern}
\ifPDFTeX\else
  % xetex/luatex font selection
\fi
% Use upquote if available, for straight quotes in verbatim environments
\IfFileExists{upquote.sty}{\usepackage{upquote}}{}
\IfFileExists{microtype.sty}{% use microtype if available
  \usepackage[]{microtype}
  \UseMicrotypeSet[protrusion]{basicmath} % disable protrusion for tt fonts
}{}
\makeatletter
\@ifundefined{KOMAClassName}{% if non-KOMA class
  \IfFileExists{parskip.sty}{%
    \usepackage{parskip}
  }{% else
    \setlength{\parindent}{0pt}
    \setlength{\parskip}{6pt plus 2pt minus 1pt}}
}{% if KOMA class
  \KOMAoptions{parskip=half}}
\makeatother
\usepackage{xcolor}
\usepackage[left=3cm,right=3cm,top=2cm,bottom=4cm]{geometry}
\usepackage{graphicx}
\makeatletter
\def\maxwidth{\ifdim\Gin@nat@width>\linewidth\linewidth\else\Gin@nat@width\fi}
\def\maxheight{\ifdim\Gin@nat@height>\textheight\textheight\else\Gin@nat@height\fi}
\makeatother
% Scale images if necessary, so that they will not overflow the page
% margins by default, and it is still possible to overwrite the defaults
% using explicit options in \includegraphics[width, height, ...]{}
\setkeys{Gin}{width=\maxwidth,height=\maxheight,keepaspectratio}
% Set default figure placement to htbp
\makeatletter
\def\fps@figure{htbp}
\makeatother
\setlength{\emergencystretch}{3em} % prevent overfull lines
\providecommand{\tightlist}{%
  \setlength{\itemsep}{0pt}\setlength{\parskip}{0pt}}
\setcounter{secnumdepth}{5}
\usepackage[T1]{fontenc} \usepackage{babel} \usepackage{float} \floatplacement{figure}{H} \usepackage{array} \usepackage{makecell} \usepackage{fancyhdr} \usepackage{graphicx} \usepackage{longtable} \usepackage{hyperref} \usepackage{fontspec} \usepackage{booktabs} \setmainfont[ BoldFont={arialbd.ttf}, ItalicFont={ariali.ttf}, BoldItalicFont={arialbi.ttf} ]{arial.ttf} \usepackage{caption} \captionsetup[table]{skip=10pt} \usepackage{titling} \usepackage{tabularx} \usepackage{titlesec} \setlength{\headheight}{60pt}
\ifLuaTeX
  \usepackage{selnolig}  % disable illegal ligatures
\fi
\usepackage{bookmark}
\IfFileExists{xurl.sty}{\usepackage{xurl}}{} % add URL line breaks if available
\urlstyle{same}
\hypersetup{
  pdfauthor={Paula Amado y David Acero},
  hidelinks,
  pdfcreator={LaTeX via pandoc}}

\title{\textbf{Taller 3}}
\author{Paula Amado y David Acero}
\date{}

\begin{document}
\maketitle

\section{Introducción}

Una nueva start-up en el sector inmobiliario busca establecer una
estrategia sólida para la compra y venta de propiedades en Chapinero,
Bogotá. Su principal objetivo es adquirir la mayor cantidad de inmuebles
en esta zona estratégica mientras optimiza los costos y minimiza los
riesgos. Para ello, han solicitado el desarrollo de un modelo predictivo
que permita tomar decisiones fundamentadas y maximizar el rendimiento de
sus inversiones.

Uno de los desafíos más significativos es la falta de información
detallada sobre las propiedades en Chapinero, lo que dificulta la
estimación precisa de precios y la definición de estrategias efectivas.
Además, la empresa está alerta ante los riesgos asociados al uso de
modelos predictivos en el sector, recordando casos como el de Zillow,
donde la sobreestimación de precios derivó en pérdidas millonarias y un
impacto negativo en su operación. Este proyecto busca superar tales
desafíos mediante el desarrollo de un modelo robusto, adaptado a las
limitaciones de datos, que garantice predicciones confiables y facilite
una toma de decisiones eficiente y sostenible.

La literatura existente subraya la relevancia de aplicar enfoques
avanzados de machine learning para mejorar la precisión en la predicción
de precios de vivienda. Por ejemplo, Truong et al.~(2020) destacaron
cómo técnicas como la regresión híbrida y la generalización apilada
combinan estrategias como Random Forest, XGBoost y LightGBM para manejar
datos complejos, reducir errores y equilibrar el sesgo y la varianza en
los modelos. Estas metodologías resultan clave para prevenir errores
costosos y asegurar resultados sostenibles.

Por su parte, Adetunji et al.~(2022) demostraron que el algoritmo Random
Forest es particularmente eficaz para predecir precios de viviendas al
considerar factores como ubicación, tamaño y condiciones físicas.
Asimismo, investigaciones como las de Varma et al.~(2018) han explorado
las ventajas de combinar diferentes métodos de predicción para obtener
estimaciones más precisas. Inspirados en estos avances, este proyecto
aprovechará los conocimientos existentes para construir un modelo
predictivo que se ajuste de manera óptima al contexto de Chapinero y sus
particularidades.

\section{Datos}
\section{Modelos y resultados} 
\section{Conclusiones y recomendaciones}
\section{Referencias}

\begin{itemize}
  \item Adetunji, A. B., Akande, O. N., Ajala, F. A., Oyewo, O., Akande, Y. F., & Oluwadara, G. (2021). House price prediction using random forest machine learning technique. Procedia Computer Science, 199, 806–813. 
  \item Varma, A., Sarma, A., Doshi, S., & Nair, R. (2018). House price prediction using machine learning and neural networks. 
  \item Truong, Q., Nguyen, M., Dang, H., & Mei, B. (2020). Housing price prediction via improved machine learning techniques. Procedia Computer Science, 174, 433–442. 
\end{itemize}

\end{document}
